\documentclass[letterpaper]{article}

\usepackage{amsmath}
\usepackage{minted}
\usepackage[ruled,vlined]{algorithm2e}
\usepackage[utf8]{inputenc}
\usepackage[english]{babel}
\usepackage{blindtext}
\usepackage{amssymb}
\usepackage{scrextend}
\usepackage{multicol}
\usepackage{wrapfig}
\usepackage{tikz}
\usepackage{listings}
\usepackage{amsthm}
\usepackage{multicol}
\usepackage{fancyhdr}
\usepackage{parskip}
\usepackage[margin=0.6in]{geometry}


% Turn on the style
\pagestyle{fancy}
% Clear the header and footer
\fancyhead{}
\fancyfoot{}
% Set the right side of the footer to be the page number
\fancyfoot[R]{\thepage}
% Redefine plain style, which is used for titlepage and chapter beginnings
% From https://tex.stackexchange.com/a/30230/828
\fancypagestyle{plain}{%
    \renewcommand{\headrulewidth}{0pt}%
    \fancyhf{}%
    \fancyfoot[R]{\thepage}%
}


\title{\vspace{-2cm}Homework 9}
\author{Yutong Huang (yxh589)}
\date{}


\begin{document}
\maketitle
\section*{Problem 1}
To prove that a language is NP-complete, we need to prove that it is NP and that every language L can be reduced to it in polynomial time.
\begin{proof}
    Assume $P=NP$, then $\forall A \in P$ such that $A \neq \phi\ \land A\neq \Sigma^* \implies A \in NP$. Then $\exists x \in L \land b\notin L$\\
    Let $L$ be an arbitrary language from $NP=P$ such that $L \neq \phi\ \land L\neq \Sigma^*$, and there must a decider $D_L$ that decides $L$ in polynomial time.\\
    Consider the following algorithm:
    \begin{verbatim}
        function reduction(w: string){
            run D_L on input w;
            if (D_L accepts){
                return x;
            }else {
                return y;
            }
        }
    \end{verbatim}
    This algorithm maps words from $L$ to words from $A$ in polynomial time.\\
    Case 1: $w \in L \implies D_L$ accepts $\implies$ \verb#reduction# returns $x \in A$ in polynomial time;\\
    Case 2: $w \notin L \implies D_L$ rejects $\implies$ \verb#reduction# returns $y \notin A$ in polynomial time;\\
    $\therefore \forall L, L \leq_P A \land A\in NP \implies A$ is NP-complete.
\end{proof}

\section*{Problem 2}
We need to prove:
\begin{enumerate}
    \item $LPATH \in NP$
    \item $\forall L \in NP, L \leq_P LPATH$
\end{enumerate}
\begin{proof}
    $LPATH \in NP$:\\
    Build a verifier for $LPATH$ that runs in polynomial time:\\
    Consider the following algorithm $V(w,c)$, where $w \in \langle G,a,b,k\rangle$ and $c$ is a path:
    \begin{verbatim}
        function V(w: <G,a,b,k>, c: path){
            if (c does not contain duplicate nodes && every node in c is also in G){
                if (c.first == a && c.last == b){
                    if (c.length >= k){
                        return true;
                    } else {
                        return false;
                    }
                } else {
                    return false;
                }
            }else {
                return false;
            }
        }
    \end{verbatim}
    This algorithm will run in $O(CW)$ time, where $C = |c|,\ W=|w|$, which is polynomial.
\end{proof}
\begin{proof}
    $\forall L \in NP, L \leq_P LPATH$\\
    Consider the $UHAMPATH = \{ \langle G,s, t \rangle | $ G is an undirected graph, s and
    t are two distinct vertices, and there is a path from s to t in
    G that passes through each vertex of G exactly once$ \}$, which is NP-complete. Then we need to show that
    $UHAMPATH \leq_P LAPTH$.\\
    Consider the following reduction:
    \begin{verbatim}
        function reduction(G: Graph, s: Vertex, t: Vertex){
            let k <- G.vertices.size() - 1;
            return <G, s, t, k>;
        }
    \end{verbatim}
    If $G = \langle V, E \rangle$, then this mapping takes $O(|V| + |E|)$ to complete.\\
    Assume that $\langle G, s, t \rangle \in UHAMPATH$,\\
    $\implies$ there is path in $G$ from $s$ to $t$ that passes through each vertex of $G$ exactly once.\\
    then if $k = |V| - 1$, $\langle G,s,t,k \rangle \in LAPTH$.\\
    Assume that $\langle G,a,b,k \rangle \in LAPTH$,\\
    $\implies$ there exists a simple path between $a$ and $b$.\\
    Because $k=|V| - 1 \implies$ this path must pass through all vertices exactly once.\\
    $\implies \langle G,a,b \rangle \in UHAMPATH$\\
    $\therefore \langle G,a,b \rangle \in UHAMPATH \iff \langle G,a,b,k \rangle \in LAPTH$\\
    $\therefore UHAMPATH \leq_P LAPTH$
\end{proof}
$\therefore LPATH$ is NP-complete.

\section*{Problem 3}
\begin{enumerate}
    \item $SET-SPLITTING \in NP$ \begin{proof}
        We can verify if each $C_i \in C$ is monochromatic in $O(|C_i|)$ time, \\
        $\implies$ we can verify if $C$ contains any monochromatic sets in $O(|C|)$ time.\\
        $\therefore$ $SET-SPLITTING \in NP$.
    \end{proof}
    \item $\forall L \in NP, L \leq_P SET-SPLITTING$ \begin{proof}
        We can prove this by reducing another NP-complete language to $SET-SPLITTING$: $3SAT$.\\
        Suppose a formula $F \in 3SAT$, $F=(x_1 \lor x_2 \lor x_3) \land \dots \land (x_{n-2} \lor x_{n-1} \lor x_n)$.\\
        Create a Set $S$ and a set of its subsets $C'$ as follows:
        \begin{enumerate}
            \item For each $x_i$, create $ x_i, \overline{x_i}$;
            \item Then create a set S=$\{x_1, \overline{x_2}, x_2, \overline{x_2},\dots x_i, \overline{x_i}\dots x_n, \overline{x_n}\}$;
            \item create a variable $f=false$; 
            \item for every clause $C_i$ in $F$, create a set of 4 variables $C_i'$ containing elementes from $C_i$ and $f$
        \end{enumerate}
        Now we color every true variale in blue and every false one in red.
        If $F$ is satisfiable, then every clause $C_i$ in $F$ must have at least one blue variable. This means that every $C_i'$ must have at least one of either color.\\
        $\therefore \langle S, C' \rangle \in SET-SPLITTING$;

        If $F$ is not satisfiable, then at least on of the clause $C_i$ in $F$ evaluates to false. Then we can find these clauses and for each one of them replace one of the $x_i$ with $\overline{x_i}$ to create a new set of clauses $C''$.\\
        $\therefore \langle S, C'' \rangle \in SET-SPLITTING$

        $\therefore 3SAT \leq_P SET-SPLITTING$ 

        $\therefore \forall L \in NP, L \leq_P SET-SPLITTING$
    \end{proof}
\end{enumerate}

\end{document}

