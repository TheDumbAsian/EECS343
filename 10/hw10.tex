\documentclass[letterpaper]{article}

\usepackage{amsmath}
\usepackage{minted}
\usepackage[ruled,vlined]{algorithm2e}
\usepackage[utf8]{inputenc}
\usepackage[english]{babel}
\usepackage{blindtext}
\usepackage{amssymb}
\usepackage{scrextend}
\usepackage{multicol}
\usepackage{wrapfig}
\usepackage{tikz}
\usepackage{listings}
\usepackage{amsthm}
\usepackage{multicol}
\usepackage{fancyhdr}
\usepackage{parskip}
\usepackage[margin=0.6in]{geometry}


% Turn on the style
\pagestyle{fancy}
% Clear the header and footer
\fancyhead{}
\fancyfoot{}
% Set the right side of the footer to be the page number
\fancyfoot[R]{\thepage}
% Redefine plain style, which is used for titlepage and chapter beginnings
% From https://tex.stackexchange.com/a/30230/828
\fancypagestyle{plain}{%
    \renewcommand{\headrulewidth}{0pt}%
    \fancyhf{}%
    \fancyfoot[R]{\thepage}%
}


\title{\vspace{-2cm}Homework 10}
\author{Yutong Huang (yxh589)}
\date{}

\begin{document}
\maketitle
\section*{Problem 1}
Need to prove: $\overline{A} \in$ \textbf{co-NP} $\land \forall L \in$ \textbf{co-NP} $L \leq_P \overline{A}$
\begin{proof}
    Assume language $A$ is NP-complete $\implies \forall L \in NP, L \leq_P A\  \land A \in NP$.\\
    Then we have $\overline{A} \in $ \textbf{co-NP} and a verifier \verb#V_a(w, c)# that runs in polynomial time.\\
    Let $B$ be an arbitrary language form \textbf{co-NP}. Then $ \overline{B} \in NP $ and $\overline{B} \leq_P A$.\\
    Then there exists a verifier \verb#V_b_complement(w, c)# that verifies $\overline{B}$ in polynomial time.\\

    The verifier for $B$ works by inverting the output of \verb#V_b_complement(w, c)#:
    \begin{verbatim}
    function V_b(w,c){
        if (V_b_complement(w,c) accepts){
            reject
        } else {
            accept
        }
    }
    \end{verbatim}
    Therefore $B \leq_P \overline{B}$. Similarly, $A \leq_P \overline{A}$.\\
    Therefore $B \leq_P \overline{B} \leq_P A \leq_P \overline{A}$.\\
    $\therefore \forall L \in$  \textbf{co-NP} $L \leq_P \overline{A}$\\

    $ \overline{A} \in $ \textbf{co-NP} $\land$ $\forall L \in$  \textbf{co-NP} $L \leq_P \overline{A} \implies \overline{A}$ is \textbf{co-NP}-complete
\end{proof}

\section*{Problem 2}
\begin{proof}
    Assume a language $L$ is NP-complete and PSPACE-complete.\\
    Therefore $\forall A \in NP, A \leq_P L$ $\land$ $\forall B \in PSPACE, B \leq_P L$\\
    Therefore $\forall A \in NP, B \in PSPACE, A \leq_P B$ and $B \leq_P A$\\
    Therefore NP = PSPACE.
\end{proof}

\section*{Problem 3}
Need to prove:
\begin{enumerate}
    \item $A_{LBA} \in$ PSPACE \begin{proof}
        
    \end{proof}
    \item $\forall L \in$ PSPACE, $L\leq_P A_{LBA}$ \begin{proof}
        
    \end{proof}
\end{enumerate}
\end{document}
