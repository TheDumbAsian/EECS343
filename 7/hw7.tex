\documentclass[letterpaper]{article}

\usepackage{amsmath}
\usepackage{minted}
\usepackage[ruled,vlined]{algorithm2e}
\usepackage[utf8]{inputenc}
\usepackage[english]{babel}
\usepackage{blindtext}
\usepackage{amssymb}
\usepackage{scrextend}
\usepackage{multicol}
\usepackage{wrapfig}
\usepackage{tikz}
\usepackage{listings}
\usepackage{amsthm}
\usepackage{multicol}
\usepackage{fancyhdr}
\usepackage{parskip}
\usepackage[margin=0.6in]{geometry}

% Turn on the style
\pagestyle{fancy}
% Clear the header and footer
\fancyhead{}
\fancyfoot{}
% Set the right side of the footer to be the page number
\fancyfoot[R]{\thepage}
% Redefine plain style, which is used for titlepage and chapter beginnings
% From https://tex.stackexchange.com/a/30230/828
\fancypagestyle{plain}{%
    \renewcommand{\headrulewidth}{0pt}%
    \fancyhf{}%
    \fancyfoot[R]{\thepage}%
}

\title{\vspace{-2cm}Homework 7}
\author{Yutong Huang (yxh589)}
\date{}

\begin{document}
\maketitle
\section*{Problem 1}
Obviously, if $K(x)$ is computable, then \verb|incompressible(n: integer)| will output every incompressible
string in $\{0,1\}^n$. To create a short description for an incompressible string using \verb|incompressible(n: integer)|,
we can create the following procedure:
\begin{verbatim}
    function description(S: string){
        string[] incompressibleList <- incompressible(|S|);
        if (S in incompressibleList):
            return S;
    }
\end{verbatim}
\begin{proof}
    Assume \verb|S| is an arbitrary incompressible string, then \verb|S| is an incompressible string
    of length $|S|$. Therefore \verb|S in incompressibleList| is true, and the procedure returns \verb|S|;\\
    $\therefore \forall s \in \{incompressible\ strings\}$, \verb|function description(S: string)| is a short description of $s$.
\end{proof}

\section*{Problem 2}
Let languages $L_1, L_2$ be arbitrary languages such that $L_1, L_2 \in P$. Then there must exist Turing machines $M_1, M_2$ that
decide $L_1$ and $L_2$ in polynomial time respectively. Assume runtimes for $M_1, M_2$ are $t_1, t_2$ respectively.\\

\textit{Union}: Consider language $L_1 \cup L_2$ and the following Turing machine:
\begin{algorithm}
    
    \caption{$M_3(w)$}
    \DontPrintSemicolon
    Run $M_1$ on $w$\;
    \If{$M_1$ accepts}{
        \textbf{Accept} and \textit{halt}\;
    }
    Run $M_2$ on $w$\;
    \If{$M_2$ accepts}{
        \textbf{Accept} and \textit{halt}\;
    }
    \textbf{Reject}\;
\end{algorithm}\\
Then $M_3$ decides $L_1 \cup L_2$ in polynomial time.
\begin{proof}
    Case 1: $w \in L_1 \cup L_2$\\
    Case 1.1: $w \in L_1 \implies M_1\ accepts \implies\ M_3$  accepts and halt; runtime is $t_1+c, c$ is constant.\\ 
    Case 1.2: $w \notin L_1 \implies M_1\ rejects \implies\ M_2\ accepts \implies M_3$  accepts and halt; runtime is $t_1+t_2+c, c$ is constant.\\ 
    Case 2: $w \notin L_1 \cup L_2 \implies M_1\ rejects \land M_2\ rejects \implies M_3$ rejects; runtime is $t_1+t_2+c, c$.\\
    $\therefore M_3$ correctly decides $L_1 \cup L_2$ in polynomial time.
\end{proof}
$\therefore L_1\cup L_2 \in P$.\\
\newpage
\textit{Concatenation}: Consider language $L_3=\{\langle ab \rangle | a \in L_1 \land b\in L_2\}$ and the following Turing machine:
\begin{algorithm}
    \caption{$M_4(w)$}
    \DontPrintSemicolon
    Initialize $M_1$ with $w$ on tape and head at the start\;
    \While{$M_1$ is not in an accept state}{
        Step through $M_1$ and copy its head movement to $M_4$\;
        \textbf{Reject} if current symbol is null\;
    }
    Initialize $M_2$ with $w$ on tape and the head position of $M_4$\;
    \While{$M_2$ is not in an accept state}{
        Step through $M_2$ and copy its head movement to $M_4$\;
        \textbf{Reject} if current symbol is null\;
    }
    \textbf{Accept}\;
\end{algorithm}\\
This machine correctly decides $L_3$ in polynomial time.
\begin{proof}
    Case 1: $w\in L_3 \implies$ both loops would run without rejecting when $M_1, M_2$ find their corresponding substring and $M_4$ would accept; runtime is $t_1+t_2+c, c$ is constance;\\
    Case 2: $w \notin L_3$\\
    Case 2.1: $w$ does not contain any substring belonging to $L_1 \implies M_4$ rejects while simulating $M_1$; runtime is $t_1+c$;\\
    Case 2.2: $w$ contains a substring that belongs to $L_1$ but no substring belonging to $L_2 \implies M_4$ rejects whilw simulating $M_2$; runtime is $t_1+t_2+c$;\\
    $\therefore M_4$ correctly decides $L_3$ in polynomial time.\\
\end{proof}
$\therefore L_3 \in P$.\\

\textit{Complement}: Consider the language $\overline{L_1}$ and the following Turing machine:
\begin{algorithm}
    \caption{$M_5(w)$}
    \DontPrintSemicolon
    Run $M_1$ on $w$\;
    \If{$M_1$ accepts}{
        \textbf{Reject} and halt\;
    }\Else{
        \textbf{Accept} and halt\;
    }
\end{algorithm}
This Turing machine correctly decides $\overline{L_1}$ in polynomial time.
\begin{proof}
    Case 1: $w \in \overline{L_1} \implies w \notin \L_1 \implies M_1$ rejects $\implies M_5$ accepts; runtime is $t1+c$;\\
    Case 2: $w \notin \overline{L_1} \implies w \in \L_1 \implies M_1$ accepts $\implies M_5$ rejects; runtime is $t1+c$;\\
    $\therefore M_5$ correctly decides $\overline{L_1}$ in polynomial time.\\
\end{proof}
$\therefore \overline{L_1} \in P$.

\section*{Problem 3}
Assume the size of $t$ is $n, n\geq 1$; then $t$ can be expressed in the form:\\
$$t=b_02^0 + b_12^1 + b_22^2 + \dots + b_{n-1}2^{n-1}$$
where $b_i \in {0,1}$. Then $q^t=q^{b_02^0 + b_12^1 + b_22^2 + \dots + b_{n-1}2^{n-1}}$.\\
Therefore, to compute $q^t$, we just need to compute each $q^{2^i}$.\\
To compute $q^{2^i}$, we first compute $q^2$ by applying $q$ on $q$, which takes $O(1)$ time.\\
Then, each $q^{2^i}$ can be computed in $O(1)$ time by applying $q^{2^{i-1}}$ on itself.\\
Therefore it takes $O(n)$ times to compute $q^t$.\\
Therefore $PERM-POWER \in P$
\end{document}
