\documentclass[letterpaper]{article}

\usepackage{amsmath}
\usepackage{minted}
\usepackage[ruled,vlined]{algorithm2e}
\usepackage[utf8]{inputenc}
\usepackage[english]{babel}
\usepackage{blindtext}
\usepackage{amssymb}
\usepackage{scrextend}
\usepackage{multicol}
\usepackage{wrapfig}
\usepackage{tikz}
\usepackage{listings}
\usepackage{amsthm}
\usepackage{multicol}
\usepackage{fancyhdr}
\usepackage{parskip}
\usepackage[margin=0.6in]{geometry}


% Turn on the style
\pagestyle{fancy}
% Clear the header and footer
\fancyhead{}
\fancyfoot{}
% Set the right side of the footer to be the page number
\fancyfoot[R]{\thepage}
% Redefine plain style, which is used for titlepage and chapter beginnings
% From https://tex.stackexchange.com/a/30230/828
\fancypagestyle{plain}{%
    \renewcommand{\headrulewidth}{0pt}%
    \fancyhf{}%
    \fancyfoot[R]{\thepage}%
}


\title{\vspace{-2cm}Homework 11}
\author{Yutong Huang (yxh589)}
\date{}

\begin{document}
\maketitle
\section*{Problem 1}
\begin{proof}
    Let $M$ be a deterministic Turing machine. Define $M$ as follows:
    \begin{enumerate}
        \item On input $\langle G \rangle$, select a vertex $u$ as the starting vertex
        \item select an edge from $u$, $(u, v)$, as the starting edge
        \item traverse the graph through $(u,v)$
        \item if $M$ comes back to $u$ through an edge other than $(u, v)$ accept, otherwise repeat step 1 to 4 until all vertices are enumerated
        \item reject.
    \end{enumerate}
    Then $M$ decides UCYCLE.\\
    Case 1: $G$ contains a cycle, then the graph traversal would eventually come across a vertex $u$ and an edge that leads back to 
    $u$ that's different from $(u,v) \implies$ $M$ accepts.\\
    Case 2: $G$ does not contain a cycle, then the traversal always comes back to $u$ via $(u,v) \implies M$ rejects.

    Since all vertices are enumerated in logspace, $M$ also decides UCYCLE in logspace.

    Therefore, UCYCLE $\in L$.
\end{proof}

\section*{Problem 2}

\makeatletter
\newcommand*{\textoverline}[1]{$\overline{\hbox{#1}}\m@th$}
\newcommand*{\textin}[1]{$\in{\hbox{#1}}\m@th$}
\makeatother
Because \textbf{NL=coNL}, we can prove that BIPARTITE $\in$ \textbf{NL} by proving that \textoverline{BIPARTITE} $\in$ \textbf{NL}.
\begin{proof}
    Let $M$ be a non deterministic Turing machine. Define $M$ as follows:
    \begin{verbatim}
        M(G: graph){
            int num <- 0;
            node start <- non deterministically choose a starting node from G;
            node next <- non deterministically choose a next node of start;
            while (num <= sizeOf(G.vertices)){
                if (next == start && num is odd){
                    accept;
                }
                next <- non deterministically choose a next node of next;
                increment num;
            }
            reject;
        }
    \end{verbatim}
    It's clear that $M$ decides \textoverline{BIPARTITE} correctly.\\ 
    Case 1: G does not have an odd cycle, then $M$ will loop through all nodes and reject;\\
    Case 2: G does have an odd cycle, then $M$ will find the node that leads back to the starting node and accept.

    And since $M$ only needs to keep track of
    \verb#num#, \verb#start# and \verb#next#, it can decides \textoverline{BIPARTITE} in logspace.

    Therefore \textoverline{BIPARTITE} $\in$ \textbf{NL} $\implies$ BIPARTITE $\in$ \textbf{NL}
\end{proof}
\newpage
\section*{Problem 3}
To prove that STRONGLY-CONNECTED is NL-complete, we need to prove that:
\begin{enumerate}
    \item STRONGLY-CONNECTED \textin{} NL \begin{proof}
        Since \textbf{NL=coNL}, we can prove this by showing that \textoverline{STRONGLY-CONNECTED} \textin{NL}.\\
        Construct a non deterministic Turing machine that decides \textoverline{STRONGLY-CONNECTED}:
        \begin{verbatim}
            function M(G: graph){
                non deterministically select 2 nodes: a, b;
                if PATH(a,b) accepts {
                    then there is a path from a to b, reject;
                } else {
                    this is not a strongly connected graph, accept;
                }
            }
        \end{verbatim}
        Since this algorithm only needs to keep track of \verb#a# and \verb#b#, it only requires log space.

        Therefore \textoverline{STRONGLY-CONNECTED} \textin{NL} $\implies$ STRONGLY-CONNECTED \textin{} NL.
    \end{proof}
    \item $\forall L \in$ NL, $L$ can be logspace reduced to STRONGLY-CONNECTED \begin{proof}
        Since PATH is NL-complete, we can prove this by reducing PATH to STRONGLY-CONNECTED.\\
        Consider the following non deterministic Turing machine:
        \begin{verbatim}
            function reduction(G: graph, s: vertex, t: vertex){
                Copy G onto output tape;
                foreach(node in G.vertices){
                    write edge(node, s) to output tape;
                    write edge(t, node) to output tape;
                }
            }
        \end{verbatim}
        Case 1: If there is a path from s to t, then the constructed graph is strongly connected because 
        now every node can get to every other node via the path from s to t.\\
        Case 2: If there is no path from s to t, then the graph is not strongly connected, because 
        the only additional edges in the constructed graph go into s and out of t, therefore no path from 
        s to t is formed.

        This procedure only needs log space to store the node pointer.

        Therefore PATH is log space reducible to STRONGLY-CONNECTED $\implies$ STRONGLY-CONNECTED is NL-complete.
    \end{proof}
\end{enumerate}
\end{document}