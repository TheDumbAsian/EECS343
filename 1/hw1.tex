\documentclass{article}

\usepackage{amsmath}
\usepackage{minted}
\usepackage[ruled,vlined]{algorithm2e}
\usepackage[utf8]{inputenc}
\usepackage[english]{babel}
\usepackage{blindtext}
\usepackage{amssymb}
\usepackage{scrextend}
\usepackage{multicol}
\usepackage{wrapfig}

\usepackage{amsthm}

\usepackage[margin=1in]{geometry}
\pagestyle{empty}

\title{\vspace{-3cm}Homework 1}
\author{Yutong Huang (yxh589)}
\date{}

\begin{document}
\pagenumbering{gobble}
\maketitle
\section*{Problem 1}
\begin{proof}
    (a) Assume $L_1$ and $L_2$ are decidable, \newline
    then $\exists m_1\ that\ decides\ L_1,\ \exists m_2\ that\ decides\ L_2$. \newline
    Because $L = L_1 \times L_2$, then $\forall (a,b) \in L, a \in L_1\ \land\ b \in L_2.$ \newline
    Consider the following algorithm $M_a$:
    \begin{algorithm}
        \caption{$M_a$}
        \DontPrintSemicolon
        \SetKwInOut{Input}{Input}\SetKwInOut{Output}{Output}

        \Input{(a, b)}
        Run $m_1$ with input a\;
        \If{$m_1$ accepts}{
            Run $m_2$ with input b\;
            \If{$m_2$ accepts}{
                \Return accept\;
            }
        }
        \Return reject\;
    \end{algorithm} \newline
    Case 1: $(a,b) \in L$:
    \begin{addmargin}[1em]{0em}
        then $a \in L_1$, $b \in L_2$. \\*
        Therefore both $m_1$ and $m_2$ accept
        $\implies$ $\forall (a,b) \in L$ $M_a$ accepts $(a,b)$
    \end{addmargin}
    Case 2: $(a,b) \notin L$
    \begin{addmargin}[1em]{0em}
        Case 2.1 $a \in L_1 \land b \notin L_2$
        \begin{addmargin}[1em]{0em}
            $m1$ will accept but $m_2$ will reject\\*
            Therefore $M_a$ will reject.
        \end{addmargin}
        Case 2.2 $a \notin L_1 \land b \in L_2$
        \begin{addmargin}[1em]{0em}
            $m1$ will reject,\\*
            therefore $M_a$ will reject.
        \end{addmargin}
        Case 3.3 $a \notin L_1 \land b \notin L_2$
        \begin{addmargin}[1em]{0em}
            $m1$ will reject,\\*
            therefore $M_a$ will reject.
        \end{addmargin}
        Therefore $\forall (a, b) \notin L$, $M_a$ will reject.
    \end{addmargin}
    In conclusion, $M_a$ decides $L$. \newline

    (b) Assume $L_1$ and $L_2$ are recognizable,
    then $\exists m_1\ that\ regocnizes\ L_1,\ \exists m_2\ that\ regocnizes\ L_2$. \newline
    Because $L = L_1 \times L_2$, then $\forall (a,b) \in L, a \in L_1\ \land\ b \in L_2.$ \newline
    Consider the following algorithm $M_b$:
    \begin{algorithm}
        \caption{$M_b$}
        \DontPrintSemicolon
        \SetKwInOut{Input}{Input}\SetKwInOut{Output}{Output}

        \Input{(a, b)}
        Run $m_1$ with input a\;
        \If{$m_1$ accepts}{
            Run $m_2$ with input b\;
            \If{$m_2$ accepts}{
                \Return accept\;
            }
        }
        \Return reject\;
    \end{algorithm}  \\*
    Case 1: $(a,b) \in L$:
    \begin{addmargin}[1em]{0em}
        then $a \in L_1$, $b \in L_2$. \\*
        Therefore both $m_1$ and $m_2$ accept
        $\implies$ $\forall (a,b) \in L$ $M_b$ accepts $(a,b)$
    \end{addmargin}
    Case 2: $(a,b) \notin L$
    \begin{addmargin}[1em]{0em}
        Case 2.1 $a \in L_1 \land b \notin L_2$
        \begin{addmargin}[1em]{0em}
            $m1$ will accept but $m_2$ will reject or loop infinitely\\*
            Therefore $M_b$ will reject or loop infinitely.
        \end{addmargin}
        Case 2.2 $a \notin L_1 \land b \in L_2$
        \begin{addmargin}[1em]{0em}
            $m1$ will reject or loop infinitely,\\*
            therefore $M_b$ will reject or loop infinitely.
        \end{addmargin}
        Case 3.3 $a \notin L_1 \land b \notin L_2$
        \begin{addmargin}[1em]{0em}
            $m1$ will reject or loop infinitely,\\*
            therefore $M_b$ will reject or loop infinitely.
        \end{addmargin}
        Therefore $\forall (a, b) \notin L$, $M_b$ will reject or loop infinitely.
    \end{addmargin}
    In conclusion, $M_b$ recognizes $L$.
\end{proof}

\section*{Problem 2}
\begin{multicols}{3}
    \textbf{Turing Machine:} \\*
    $L=\{a^nb^{2n}|n \in \mathbb{N} \}$\\*
    $\Sigma = \{a,b\}$ \\*
    $\Gamma = \{a,b,\phi, x, y\}$ \\*
    $Q = \{q_1,q_2,q_3,q_4,q_5,q_6,q_{accept},q_{reject}\}$ \\*

    \columnbreak
    $\delta:$
    \begin{addmargin}[1em]{0em}
        $(q_1, a) \rightarrow (q_2, x, R)$\\*
        $(q_1, b) \rightarrow (q_{reject}\dots)$\\*
        $(q_1, x) \rightarrow (q_{reject}\dots)$\\*
        $(q_1, y) \rightarrow (q_{reject}\dots)$\\*
        $(q_1, \phi) \rightarrow (q_{accept}\dots)$\\*
        $(q_2, a) \rightarrow (q_2, a, R)$\\*
        $(q_2, b) \rightarrow (q_3, y, R)$\\*
        $(q_2, x) \rightarrow (q_2, x, R)$\\*
        $(q_2, y) \rightarrow (q_2, y, R)$\\*
        $(q_2, \phi) \rightarrow (q_{reject}\dots)$\\*
        $(q_3, a) \rightarrow (q_{reject}\dots)$\\*
        $(q_3, b) \rightarrow (q_4, y, L)$\\*
        $(q_3, x) \rightarrow (q_{reject}\dots)$\\*
        $(q_3, y) \rightarrow (q_3, y, R)$\\*
        $(q_3, \phi) \rightarrow (q_{reject}\dots)$\\*

        \columnbreak
        $(q_4, a) \rightarrow (q_4, a, L)$\\*
        $(q_4, b) \rightarrow (q_4, b, L)$\\*
        $(q_4, x) \rightarrow (q_5, x, R)$\\*
        $(q_4, y) \rightarrow (q_4, y, L)$\\*
        $(q_4, \phi) \rightarrow (q_{reject}\dots)$\\*
        $(q_5, a) \rightarrow (q_2, x, R)$\\*
        $(q_5, b) \rightarrow (q_{reject}\dots)$\\*
        $(q_5, x) \rightarrow (q_{reject}\dots)$\\*
        $(q_5, y) \rightarrow (q_6, y, R)$\\*
        $(q_5, \phi) \rightarrow (q_{reject}\dots)$\\*
        $(q_6, a) \rightarrow (q_{reject}\dots)$\\*
        $(q_6, b) \rightarrow (q_{reject}\dots)$\\*
        $(q_6, x) \rightarrow (q_{reject}\dots)$\\*
        $(q_6, y) \rightarrow (q_6, y, R\dots)$\\*
        $(q_6, \phi)\rightarrow (q_{accepts}\dots)$ \\*
    \end{addmargin}
\end{multicols}


\begin{multicols}{2}

    \begin{proof}
        Case 1: input $\in L$
        \begin{addmargin}[1em]{0em}
            Proof by induction:\\*
            Base case: n = 0 $\implies$ input is $\phi$\\*
            The machine accepts immediately.\\*
            Base case: n = 1 $\implies$ input is $abb$ \\*
            Steping through the machine: \\*
            \begin{center}
                \centering
                \begin{tabular}{ |c|c|c|c| }
                    \hline
                    Step & State        & Tape & Head   \\
                    \hline
                    0    & $q_1$        & abb  & a      \\
                    1    & $q_2$        & xbb  & b      \\
                    2    & $q_3$        & xyb  & b      \\
                    3    & $q_4$        & xyy  & y      \\
                    4    & $q_4$        & xyy  & x      \\
                    5    & $q_5$        & xyy  & y      \\
                    6    & $q_6$        & xyy  & y      \\
                    7    & $q_6$        & xyy  & $\phi$ \\
                    8    & $q_{accept}$ & xyy  & $\phi$ \\
                    \hline
                \end{tabular}
            \end{center}
            Induction: assume the machine accepts input $a^kb^{2k}, k \in \mathbb{N}$,
            $k\geq1$. \\*
            Consider input $a^kab^{2k}bb$, the machine will run through the first k groups of
            a's and b's, at state $q_3$ the tape would be $x^kay^{2k}bb$. Then the machine
            would enter $q_4$ to go back and find the last a, thus producing $x^{k+1}y^{2k}bb$.
            Next the machine would enter $q_2$ and find the second to last b, replace it with y,
            and enter $q_3$, and then replace the last b with y and enter $q_4$. This time, the
            machine will not find any a at $q_5$ and would enter $q_6$ to check if there's any leftover
            b's. When it finds that there's no b left, it will accept the input. \\*
            \\*
            Therefore, the machine accepts input $\in L$.
        \end{addmargin}
        Case 2: input $\notin L$
        \begin{addmargin}[1em]{0em}
            Case 2.1: too many a's\\*
            The machine will find a $\phi$ in either $q_3\ or\ q_4$, and reject the input.\\*
            Case 2.2: too many b's\\*
            The machine will find a b in $q_6$ and reject.\\*
            Case 2.3: a after b\\*
            The machine will find a b in $q_6$ and reject.\\*\\*
            Therefore, the machine will reject all input $\notin L$.
        \end{addmargin}
        In conclusion, the machine is correct.
    \end{proof}
\end{multicols}

\section*{Problem 3}
\begin{multicols}{3}
    \textbf{Turing Machine:} \\*
    Assume this machine can stay at a cell (S).\\*
    $L=\{a^nb^{2n}|n \in \mathbb{N} \}$\\*
    $\Sigma = \{a,b\}$ \\*
    $\Gamma = \{a,b,\_, x, y, s\}$ \\*
    $Q = \{q_1,q_2,q_3,q_4,q_5,q_6,q_7,q_8,q_9,\newline q_{accept},q_{reject}\}$ \\*

    \columnbreak
    $\delta:$
    \begin{addmargin}[1em]{0em}
        $(q_1, a) \rightarrow (q_9, s, R) $\\*
        $(q_1, b) \rightarrow (q_4, s, R) $\\*
        $(q_1, x) \rightarrow (q_{reject}\dots) $\\*
        $(q_1, y) \rightarrow (q_{reject}\dots) $\\*
        $(q_1, \_) \rightarrow (q_{accept}\dots) $\\*
        $(q_2, a) \rightarrow (q_2, a, R) $\\*
        $(q_2, b) \rightarrow (q_3, y, S) $\\*
        $(q_2, x) \rightarrow (q_2, x, R) $\\*
        $(q_2, y) \rightarrow (q_2, y, R) $\\*
        $(q_2, \_) \rightarrow (q_{reject}\dots) $\\*
        $(q_3, s) \rightarrow (q_4, \_, R) $\\*
        $(q_3, \_) \rightarrow (q_4, \_, R) $\\*
        $(q_3, *) \rightarrow (q_3, *, L) $\\*
        $(q_4, a) \rightarrow (q_5, x, R) $\\*
        $(q_4, b) \rightarrow (q_4, b, R) $\\*
        $(q_4, x) \rightarrow (q_4, x, R) $\\*
        $(q_4, y) \rightarrow (q_4, y, R) $\\*
        \columnbreak
        $(q_4, \_) \rightarrow (q_{reject}\dots) $\\*
        $(q_5, a) \rightarrow (q_6, x, S) $\\*
        $(q_5, b) \rightarrow (q_5, b, R) $\\*
        $(q_5, x) \rightarrow (q_5, x, R) $\\*
        $(q_5, y) \rightarrow (q_5, y, R) $\\*
        $(q_5, \_) \rightarrow (q_{reject}\dots) $\\*
        $(q_6, s) \rightarrow (q_7, s, R) $\\*
        $(q_6, \_) \rightarrow (q_7, \_, R) $\\*
        $(q_6, *) \rightarrow (q_6, *, L) $\\*
        $(q_7, b) \rightarrow (q_3, y, S) $\\*
        $(q_7, \_) \rightarrow (q_8, \_, L) $\\*
        $(q_7, *) \rightarrow (q_7, *, R) $\\*
        $(q_8, \_) \rightarrow (q_{accept}) $\\*
        $(q_8, a) \rightarrow (q_{reject}) $\\*
        $(q_8, s) \rightarrow (q_{accept}) $\\*
        $(q_8, *) \rightarrow (q_8, *, L) $\\*
        $(q_9, \_) \rightarrow (q_6, a, L)$\\*
        $(q_9, *) \rightarrow (q_9, *, L)$\\*
    \end{addmargin}
\end{multicols}

\begin{multicols}{2}

    \begin{proof}
        Case 1: input $\in L$
        \begin{addmargin}[1em]{0em}
            Proof by induction\\*
            Base case: Empty input:\\*
            The machine accepts immediately.\\*
            Base case: 2 'a' and 1 'b':\\*
            The machine will find 1 'b', then go back to the start (initial position of header),
            and start looking for 'a's. When it finds 2 'a's, the machine would go back to
            the start again, and check for additional 'b's. In this case, there is no more 'b',
            the machine then checks if there's any redundant 'a', which there is not in this case.
            Therefore, the machine accepts languages consists of 2 'a's and 1 'b'.\\*

            Induction Hypothesis: The machine accepts 2k 'a's and k 'b's, $k\in \mathbb{N} \land
                k \geq 2$.\\*
            Consider an input consists of 2k+2 'a's and k+1 'b's, the machine will find and mark the
            first k groups of 3 charaters (2 'a's and 1 'b'). Then there's only 2 'a's and 1 'b' left.
            At this point, the machine should be in state $q_7$ at the start of the input, looking for additional 'b's. It will
            find the 'b' and 2 of its corresponding 'a's. After that, with no more 'b's,
            the machine would be in state $q_8$ looking for redundant 'a's. It will not find
            any in this case, therefore it will accept the input.

            Therefore, the machine accepts all inputs $\in L$.\\*
        \end{addmargin}
        Case 2: input $\notin L$
        \begin{addmargin}[1em]{0em}
            Case 2.1: too many a's\\*
            The machine will find redundant 'a's in $q_8$ and reject.\\*
            Case 2.2: too many b's\\*
            The machine will find a \_ in either $q_4$ or $q_5$ and reject
        \end{addmargin}
        Therefore, the machine will reject all inputs $\notin L$\\*\\*

        In conclusion, the machine is correct.
    \end{proof}
\end{multicols}

\end{document}


