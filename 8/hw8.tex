\documentclass[letterpaper]{article}

\usepackage{amsmath}
\usepackage{minted}
\usepackage[ruled,vlined]{algorithm2e}
\usepackage[utf8]{inputenc}
\usepackage[english]{babel}
\usepackage{blindtext}
\usepackage{amssymb}
\usepackage{scrextend}
\usepackage{multicol}
\usepackage{wrapfig}
\usepackage{tikz}
\usepackage{listings}
\usepackage{amsthm}
\usepackage{multicol}
\usepackage{fancyhdr}
\usepackage{parskip}
\usepackage[margin=0.6in]{geometry}


% Turn on the style
\pagestyle{fancy}
% Clear the header and footer
\fancyhead{}
\fancyfoot{}
% Set the right side of the footer to be the page number
\fancyfoot[R]{\thepage}
% Redefine plain style, which is used for titlepage and chapter beginnings
% From https://tex.stackexchange.com/a/30230/828
\fancypagestyle{plain}{%
    \renewcommand{\headrulewidth}{0pt}%
    \fancyhf{}%
    \fancyfoot[R]{\thepage}%
}


\title{\vspace{-2cm}Homework 8}
\author{Yutong Huang (yxh589)}
\date{}

\begin{document}
\maketitle

\section*{Problem 1}
Assume languages $L_1, L_2 \in NP \iff \exists$ deterministic TMs, $V_1(x, y)$ and $V_2(x, y)$, and polynomials $p_1, p_2, q_1, q_2$ such that\\
$\forall x, y$, $V_1$ runs in $p_1(|x|)$ time and $V_2$ runs in $p_2(|x|)$ time $\land$\\
$\forall x \in L_1, \exists y$ such that $|y|=q_1(|x|), V_1(x, y) accepts$ and $\forall x \in L_2, \exists y$ such that $|y|=q_2(|x|), V_2(x, y) accepts\land$\\
$\forall x \notin L_1, \forall y \in \{y \mid |y| = p_1(|x|)\}, V_1(x, y)\ rejects$ and $\forall x \notin L_2, \forall y \in \{y \mid |y| = p_2(|x|)\}, V_2(x, y)\ rejects$.
\subsection*{union -- $L_1 \cup L_2 \in NP$}
\begin{proof}
    Construct a verifier TM $V_3$:
    \begin{verbatim}
    function v3(x,y) {
        run v1(x,y);
        if v1 accepts
            accept;
        run v2(x,y)
        if v2 accepts
            accept;
        reject;
    }
\end{verbatim}
    We know that by definition, $V_1$ and $V_2$ runs in polynomial times, therefore $V_3$ runs in polynomial time as well.\\

    Case 1: $x\in L_1 \implies \exists y \in \{y||y| = q_1(|x|)\} such\ that\ V_1\ accepts$ $\implies \exists y \in \{y||y| = q_1(|x|)\} such\ that\ V_3\ accepts$\\
    Case 2: $x\in L_1$\\
    Case 2.1: $x \in L_2\implies \exists y \in \{y||y| = q_2(|x|)\} such\ that\ V_2\ accepts$ $\implies \exists y \in \{y||y| = q_2(|x|)\} such\ that\ V_3\ accepts$\\
    Case 2.2: $x \notin L_2 \implies$ both $V_1$ and $V_2$ reject $\implies\ V_3$ rejects.\\

    Therefore $V_3$ verifies if $x\in L$ with advice string y in polynomial time $\iff L_1 \cup L_2 \in NP$.
\end{proof}

\subsection*{concatenation -- $L_1 \circ L_2 \in NP$}
\begin{proof}
    Construct a verifier TM $V_4$:
    \begin{verbatim}
    function v4(x, y) {
        if (y is in the form "k#a#b"){ //k is a number, a and b are strings, # is a new character
            run v1(x[0:k-1], a); //slicing operator represents substrings
            run v2(x[k:], b);
            if (v1 accepts) and (v2 accept){
                accept;
            }
            reject;
        }
        reject;
    }
\end{verbatim}

    Similarly, because both $V_1$ and $V_2$ run in polynomial times, $V_4$ also runs in polynomial time.\\

    Also by definition, we know that if a verifier accepts, then the advice string length is polynomial of the input string.\\

    Case 1: $x\in L_1 \circ L_2 \implies a,$ and $b$ are polynomials of $x_1x_2\dots x_k$ and $x_{k+1}\dots x_n \implies$ "k\#a\#b" is polynomial of $x$.
    Also $V_1, V_2$ accpet in this case $\implies$ $V_4$ accepts. Therefore $\forall x \in L_1 \circ L_2, \exists y \in {y||y|=p(|x|)} \land V_4$ accepts.\\
    Case 2: $x\notin L_1 \circ L_2 \implies $ at least one of $V_1, V_2$ rejects $\implies$ for rejecting machine $V_i$: $\forall x\notin L_i, \forall y \in {y||y|=p(|x|)} \land V_i$ rejects.
    $\implies$ $\forall x\notin L_1 \circ L_2, \forall y \in {y||y|=p(|x|)} \land V_4$ rejects.

    Therefore, $L_1 \circ L_2 \in NP$.
\end{proof}

\section*{Problem 2}
Because the Clique Problem is NP-Complete, we can try to reduce the Clique Problem to SUBGRAPH ISOMORPHISM to show that it is in NP.\\

\textbf{Claim: $CLIQUE \leq_{P} SUBGRAPH\ ISOMORPHISM$}
\begin{proof}
    Suppose there is an algorithm \verb#sgi(g: Graph, h: Graph)# that decides if \verb#h# is isomorphic to a subgraph of \verb#g#.\\
    Consider the following algorithm that decides if a graph has a clique with a given number of vertices:
    \begin{verbatim}
        function clique(g: Graph, k: int){
            generate a complete graph h with k vertices;
            return sgi(g, h);
        }
    \end{verbatim}
    Case 1: If \verb#g# has a clique of \verb#k# vertices, then \verb#g# has a subgraph that's isomorphic to \verb#h#, meaning that \verb#sgi(g, h)# returns true, and thus \verb#clique(g, k)# is true;\\
    Case 2: If \verb#g# does not have a clique of \verb#k# verices, then \verb#g# does not have a subgraph that's isomorphic to \verb#h#, meaning that \verb#sgi(g,h)# returns false, and thus \verb#clique(g,k)# returns false;\\

    Therefore, the algorithm is correct.\\

    Assume the input graph has \verb#n# vertices, the input size could be as large as $n^2+n+1$. The generation of the complete graph \verb#h# can be performed in $O(k^2)$, which is smaller than $O(n^2)$.\\

    Therefore, this reduction runs in polynomial times $\implies CLIQUE \leq_{P} SUBGRAPH\ ISOMORPHISM$.\\

    Since $CLIQUE$ is NP-complete, then $SUBGRAPH\ ISOMORPHISM \in NP$.

\end{proof}

\section*{Problem 3}

\end{document}