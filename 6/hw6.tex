\documentclass[letterpaper]{article}

\usepackage{amsmath}
\usepackage{minted}
\usepackage[ruled,vlined]{algorithm2e}
\usepackage[utf8]{inputenc}
\usepackage[english]{babel}
\usepackage{blindtext}
\usepackage{amssymb}
\usepackage{scrextend}
\usepackage{multicol}
\usepackage{wrapfig}
\usepackage{tikz}
\usepackage{listings}
\usepackage{amsthm}
\usepackage{multicol}
\usepackage{fancyhdr}
\usepackage{parskip}
\usepackage[margin=0.6in]{geometry}

% Turn on the style
\pagestyle{fancy}
% Clear the header and footer
\fancyhead{}
\fancyfoot{}
% Set the right side of the footer to be the page number
\fancyfoot[R]{\thepage}
% Redefine plain style, which is used for titlepage and chapter beginnings
% From https://tex.stackexchange.com/a/30230/828
\fancypagestyle{plain}{%
    \renewcommand{\headrulewidth}{0pt}%
    \fancyhf{}%
    \fancyfoot[R]{\thepage}%
}

\title{\vspace{-2cm}Homework 6}
\author{Yutong Huang (yxh589)}
\date{}


\begin{document}
\maketitle
\section*{Problem 1}

\begin{multicols}{2}
    \begin{minipage}{0.4\textwidth}
        \begin{algorithm}[H]
            \caption{$T(\langle M, w\rangle)$}
            \Return $\langle M \rangle$
        \end{algorithm}
        According to recursion theorem, $\exists r(w) \rightarrow \langle r(w) \rangle $

    \end{minipage}

    \begin{minipage}{0.4\textwidth}
        \begin{algorithm}[H]
            \caption{$M(w)$}
            Obtain $ \langle M \rangle $ by invoking $ T(\langle M, w\rangle) $\;
            Print $ \langle r( \langle M \rangle ) \rangle $ and halt\;
        \end{algorithm}
        $ N = r( \langle M \rangle ) $;\\
        $ \therefore  M $ prints $ \langle N \rangle $, and $N$ prints $ \langle M \rangle $
    \end{minipage}

\end{multicols}

\section*{Problem 2}
The following java program is submitted through canvas. A and B are implemented in such a way that they print each other.
\begin{minted}{java}
/**
 * Quine - a class with two method A and B such that the class prints the source code of itself, 
 *          and A and B print each other.
 */
public class Quine {

    /**
     * This method prints the start of the class, start of the main function, 
     * the string that contains the partial source code, and then the rest of this method; <p>
     * Then this method invokes B() to print method A, and then invokes A() to print B; <p>
     * And finally print the close brackets.
     * @param args - no args needed
     */
    public static void main(String[] args) {
        String[] sourceOfMain = {
            "public class Quine {",
            "   public static void main(String[] args) {",
            "       String[] sourceOfMain = {",
            "           ",
            "       }",
            "       char q = 34;",
            "       for (int i = 0; i < 3; i++) {",
            "           System.out.println(sourceOfMain[i]);",
            "       }",
            "       for (String line : sourceOfMain) {",
            "           System.out.println(sourceOfMain[3] + q + line + q);",
            "       }",
            "       for (int i = 3; i < 21; i++) {",
            "           System.out.println(sourceOfMain[i]);",
            "       }",
            "       B();",
            "       A();",
            "       for (int i = 21; i < sourceOfMain.length; i++) {",
            "           System.out.println(sourceOfMain[i]);",
            "       }",
            "   }",
            "   ",
            "}"
        };

        char q = 34;
        for (int i = 0; i < 3; i++) {
            System.out.println(sourceOfMain[i]);
        }
        for (String line : sourceOfMain) {
            System.out.println(sourceOfMain[3] + q + line + q);
        }
        for (int i = 3; i < 21; i++) {
            System.out.println(sourceOfMain[i]);
        }
        B();
        A();
        for (int i = 21; i < sourceOfMain.length; i++) {
            System.out.println(sourceOfMain[i]);
        }

    }

    /**
     * Print method {@code Quine.B}
     */
    public static void A(){

        String[] sourceOfA = {
            "   public static void A(){",
            "      String sourceOfA = {",
            "          ",
            "      };",
            "      String sourceOfB = {",
            "          ",
            "      };",
            "      char q = 34;",
            "      for (int i = 0; i < 2; i++) {",
            "          System.out.println(sourceOfB[i]);",
            "      }",
            "      for (String line : sourceOfA) {",
            "          System.out.println(sourceOfB[2] + q + line + q);",
            "      }",
            "      for (int i = 3; i < 5; i++) {",
            "          System.out.println(sourceOfB[i]);",
            "      }",
            "      for (String line : sourceOfB) {",
            "          System.out.println(sourceOfB[5] + q + line + q);",
            "      }",
            "      for (int i = 6; i < sourceOfB.length; i++) {",
            "          System.out.println(sourceOfB[i]);",
            "      }",
            "   }"
        };

        String[] sourceOfB = {
            "   public static void B(){",
            "      String sourceOfA = {",
            "          ",
            "      };",
            "      String sourceOfB = {",
            "          ",
            "      };",
            "      char q = 34;",
            "      for (int i = 0; i < 2; i++) {",
            "          System.out.println(sourceOfA[i]);",
            "      }",
            "      for (String line : sourceOfA) {",
            "          System.out.println(sourceOfA[2] + q + line + q);",
            "      }",
            "      for (int i = 3; i < 5; i++) {",
            "          System.out.println(sourceOfA[i]);",
            "      }",
            "      for (String line : sourceOfB) {",
            "          System.out.println(sourceOfA[5] + q + line + q);",
            "      }",
            "      for (int i = 6; i < sourceOfB.length; i++) {",
            "          System.out.println(sourceOfA[i]);",
            "      }",
            "   }"
        };

        char q = 34;
        for (int i = 0; i < 2; i++) {
            System.out.println(sourceOfB[i]);
        }
        for (String line : sourceOfA) {
            System.out.println(sourceOfB[2] + q + line + q);
        }
        for (int i = 3; i < 5; i++) {
            System.out.println(sourceOfB[i]);
        }
        for (String line : sourceOfB) {
            System.out.println(sourceOfB[5] + q + line + q);
        }
        for (int i = 6; i < sourceOfB.length; i++) {
            System.out.println(sourceOfB[i]);
        }
    }

    /**
     * Print method {@code Quine.A}
     */
    public static void B(){

        String[] sourceOfA = {
            "   public static void A(){",
            "      String sourceOfA = {",
            "          ",
            "      };",
            "      String sourceOfB = {",
            "          ",
            "      };",
            "      char q = 34;",
            "      for (int i = 0; i < 2; i++) {",
            "          System.out.println(sourceOfB[i]);",
            "      }",
            "      for (String line : sourceOfA) {",
            "          System.out.println(sourceOfB[2] + q + line + q);",
            "      }",
            "      for (int i = 3; i < 5; i++) {",
            "          System.out.println(sourceOfB[i]);",
            "      }",
            "      for (String line : sourceOfB) {",
            "          System.out.println(sourceOfB[5] + q + line + q);",
            "      }",
            "      for (int i = 6; i < sourceOfB.length; i++) {",
            "          System.out.println(sourceOfB[i]);",
            "      }",
            "   }"
        };

        String[] sourceOfB = {
            "   public static void B(){",
            "      String sourceOfA = {",
            "          ",
            "      };",
            "      String sourceOfB = {",
            "          ",
            "      };",
            "      char q = 34;",
            "      for (int i = 0; i < 2; i++) {",
            "          System.out.println(sourceOfA[i]);",
            "      }",
            "      for (String line : sourceOfA) {",
            "          System.out.println(sourceOfA[2] + q + line + q);",
            "      }",
            "      for (int i = 3; i < 5; i++) {",
            "          System.out.println(sourceOfA[i]);",
            "      }",
            "      for (String line : sourceOfB) {",
            "          System.out.println(sourceOfA[5] + q + line + q);",
            "      }",
            "      for (int i = 6; i < sourceOfB.length; i++) {",
            "          System.out.println(sourceOfA[i]);",
            "      }",
            "   }"
        };
        
        char q = 34;
        for (int i = 0; i < 2; i++) {
            System.out.println(sourceOfA[i]);
        }
        for (String line : sourceOfA) {
            System.out.println(sourceOfA[2] + q + line + q);
        }
        for (int i = 3; i < 5; i++) {
            System.out.println(sourceOfA[i]);
        }
        for (String line : sourceOfB) {
            System.out.println(sourceOfA[5] + q + line + q);
        }
        for (int i = 6; i < sourceOfB.length; i++) {
            System.out.println(sourceOfA[i]);
        }
    }
}
\end{minted}
\newpage
\section*{Problem 3}

In class we proved that $ Th(\mathbb{N}, +, \times) $ is undecidable
by showing that there are some sentences in $ Th $ that are true but
unprovable. This is done by constructing a Turing machine $S$ with a hypothetical
true but unprovable sentence $\varphi = \exists c [\phi_{S,0}] $ such that $S$ only accepts if $ \lnot \varphi $
is proved to be true.\\

This fails at $F_{m}$ because $F_{m}$ is decidable, and thus contains no sentences that are true but unprovable.
\begin{proof}
It is obvious that $Z_m$ is finite, therefore we can simply enumerate all the possible
values to check if the sentence holds. Consider the following procedure:
\begin{enumerate}
    \item let the sentence be: $ \exists x_i[\phi_i(x_1,x_2\dots x_i)] $;
    \item iterate $i$ from $ 0 $ to $ m-1 $ and check if the sentence holds;
    \item if all iterations of 2 shows the sentence holds, then the it is true, otherwise it is false.
\end{enumerate}
The above procedure correctly decides $F_m$\\
$ \therefore F_m $ is decidable.
\end{proof}
\end{document}